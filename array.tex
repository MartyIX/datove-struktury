% Součást skript na Datové struktury. Viz main.tex
\markright{$ $Id$ $}

\chapter{Uspořádaná pole}

% --------------------------------------------------------------------------
\section{Unární, binární a interpolační vyhledávání}
% \mnote{Napsal Pavel Machek}
% pavel@ucw.cz

Uspořádané pole je datová struktura, která vznikne z pole jeho
setříděním. Jediná operace, která se na ní dá (rozumně rychle)
provádět, je MEMBER.

Mějme slovník $S$ uložený jako pole prvků tak, že $s[i] < s[i+1]$.

\begin{algorithm}
\caption{MEMBER pro uspořádané pole}
\label{alg:bin.member}
\begin{algorithmic}
\STATE \COMMENT {vyhledání hodnoty $x$ mezi $s[i] \dots s[j]$}
\STATE \COMMENT {odpověď ANO, když $\exists h : i \leq h \leq j \land s[h]=x$}
\STATE $d := i$ \COMMENT {aktuální dolní a horní odhad}
\STATE $h := j$
\STATE $\<next> := f(d, h)$
	\COMMENT { Předpokládáme, že $d \leq f(d,h) \leq h$ }
\WHILE {$s[\<next>] \neq x \land d < h$}
	\IF {$s[\<next>] < x$}
		\STATE $d := \<next> + 1$
	\ELSE
		\STATE $h := \<next> - 1$
	\ENDIF
        \STATE $\<next> := f(d, h)$
\ENDWHILE
\STATE \COMMENT {řekni ANO pokud $s[\<next>] = x$, jinak řekni ne}
\end{algorithmic}
\end{algorithm}

Tento algoritmus může provádět unární, binární, nebo interpolační vyhledávání;
stačí jen dosadit správnou funkci $f$; 
zobecněné kvadratické vyhledávání bude definováno dále.

\hspace{10mm}

\begin{tabular}{|l|l|l|l|}
\hline
\bf{metoda}& \bf{odpovídající funkce}& nejhorší př.& průměrný případ\\
\hline
	unární vyhledávání& 
$f(d,h) = d$&
$O(n)$&
$O(n)$\\
	binární vyhledávání&
$f(d,h) = \lceil \frac{d+h}{2} \rceil$&
$O(\log(n))$&
$O(\log(n))$\\
	interpolační vyhledávání&
$f(d,h) = d + \lceil \frac{x-s[d]}{s[h]-s[d]} * (h-d+1) \rceil$ &
$O(n)$&
$O(\log(\log(n)))$\\
\hline
	zobecněné kvadratické v.&
$f(d,h) = fkvadrat$&
$O(\log(n))$&
$O(\log(\log(n)))$\\
	kvadratické vyhledávání&
$ $&
$O(\log(n))$&
$O(\log(\log(n)))$\\
\hline
\end{tabular}

\mnote{Z těch zápisků, co mám, to opravdu vypadá, 
jako že zobecněné kvadratické a kvadratické jsou 2 různé věci}

\section{Zobecněné kvadratické vyhledávání}
% \mnote{algoritmus vychází z Pavlova,
% výklad napsal Jakub Černý} 
% kuba@atrey.karlin.mff.cuni.cz

\def\xx{
Poznamka pro Martina:

Tak jak to dělal Koubek na přednášce. Jinak to tvoje zobecněné kv. v.
a kvadratické vyhledávání je to samé. Pavel Machek jen přejmenoval
některé proměnné a napsal to efektivneji, ale zase mene nazorneji a
je to vice obtizne na pochopeni. (je to jak v Ccku) Na druhou stranu
je to pekna ukazka, jak neco hezky naprogramovat. 
Ten jeho popis a zduvodneni casove slozitosti je nepresny - popisuje a
upocitava jiny jednodusi alg (kod ma spravny). Bylo by zajimave zjistit
jak se tyto dva alg. chovaji. Ten alg. co popisoval Koubek je taky dost 
podobny algoritmu jump search, kde se skace po odmocninach z n a uvadi se 
ze je dost rychly.
}


% XXX tady by to melo rozdelit nejake slovo
Na interpolačním vyhledávání se nám líbí jeho čas $O(\log\log |S|)$
v~průměrném případě (při rovnoměrném rozdělení dat). Avšak jeho čas 
v~nejhorším případě je až $O(|S|)$. Zato binární vyhledávání má čas
nejvýše $O(\log|S|)$. Zobecněné kvadratické vyhledávání je tak trochu 
kombinace předchozích dvou vyhledávání.


Jak zobecněné kvadratické vyhledávání funguje? 
Využívá funkci MEMBER s funkcí {\it fkvadrat} tak, jak byla popsána 
v předchozím odstavci. Tomu, že se zvolí hodnota $next$ a podle ní se 
opraví hodnota $d$ nebo $h$, budeme říkat, že se položí dotaz. 
Celé vyhledávání funguje tak, že se nejprve položí interpolační dotaz. 
To je vždy, když je \<nastav> true.
Položení dalších dotazů si můžeme představovat jako skoky z místa posledního dotazu
ve směru, kde leží $x$. (Skočíme na nový index v poli).
\footnote{
 zde by byl vhodný obrázek - usečka, která má na krajich d a h a je
 na ni videt prvni interpolačni dotaz a skoky po sqrt(n) a bin. a
 sqrt(n) ...
}
Po interpolačním dotazu se neustále střídají skoky o $\sqrt{delka}$ 
s binárními dotazy, až dokud nepřeskočíme $x$. 
(Toto střídání zajištuje proměnná $parita$).
Pak se znova položí interpolační dotaz a vše se opakuje.

\begin{algorithm}
\caption{Krok zobecněného kvadratického vyhledávání --- $fkvadrat(d,h)$}
\label{alg:kvadratic}
\begin{algorithmic}
\STATE \COMMENT {Proměnné \<nastav>, \<parita> a \<nahoru> jsou statické, tj.
		jejich hodnoty se mezi voláními tohoto algoritmu zachovávají.}
\STATE \COMMENT {Nechť \<nastav> je na začátku \<true>.}
\STATE \COMMENT {Dokud je \<nastav> \<false> (pracuje se v rámci bloku), je 
		\<parita> střídavě \<true> (skok o $\sqrt{\<delka>}$)
		a \<false> (binární vyhledání)}
\IF {\<nastav>}
	\STATE $\<parita> := true$
	\STATE $\<delka> := h-d+1$
	\STATE $\<next> := d + \left\lceil \frac{ x-s[d] }{ s[h]-s[d] } 
						\cdot \<delka> \right\rceil$
	\COMMENT {$ = finterp(d,h)$}
	\STATE $\<nahoru> := s[\<next>] < x$
	\STATE $\<nastav> := \<false>$
	\STATE return \<next>
\ENDIF
\IF {not \<parita>}
	\STATE $\<next> := \lceil (d+h)/2 \rceil$ \COMMENT {$ = fbin(d,h)$}
	\STATE $\<parita> := true$
	\STATE return \<next>
\ENDIF
\STATE $\<next> := \<nahoru>\ ?\ d+\sqrt{\<delka>} : h-\sqrt{\<delka>}$
\IF {$s[\<next>] < x$ xor $\<nahoru>$}
	\STATE $\<nastav> := true$
\ELSE
	\STATE $\<parita> := false$
\ENDIF
\STATE return $\<next>$
\end{algorithmic}
\end{algorithm}

Jaký čas má vyhledávání v nejhorším případě? Rozdíl mezi $d$ a $h$ se během nejvýše 3 dotazů zmenší
na polovinu. Proto je nejhorší čas $O(\log n)$.

 
Jaký čas má vyhledávání v průměrném případě? Tím myslíme při rovnoměrném 
rozložení dat. To už je malinko složitější otázka.
Vyhledávání si rozdělíme do několika fází. Fáze začíná interpolačním dotazem a
pokračuje až do dalšího interpolačního dotazu. Ukážeme, že v jedné fázi se
položí v průměru jen konstantně dotazů. Pojďme tedy zanalyzovat jednu fázi.
Souvislý úsek pole mezi pozicemi $d$ a $h$ na začátku fáze označme jako blok.
Proměnná $delka$ udává délku bloku a má hodnotu $h-d+1$.
Označme $X$ náhodnou proměnnou, $X=\hbox{počet $i$ na začátku bloku takových, že }i\ge
d \hbox{ a } s[i]<x$. Jinak řečeno $X$ udává vzdálenost $x$ od začátku bloku.

Položme $p=\pr(\hbox{náhodně zvolený prvek mezi $s[d]$ a $s[h]$ je
menší než }x)={(x-s[d])}/{(s[h]-s[d])}$

% |X| = j -> X = j , protoze X uz oznacuje kardinalitu
$$\pr(X=j) = \binom{h-d+1}{j} p^j (1-p)^{h-d+1-j}$$

$X$ má tedy binomické rozdělení a tudíž je jeho očekávaná hodnota
$p(h-d+1)$ a jeho rozptyl je $p(1-p)(h-d+1)$.
Označme $prv$ pozici v vrácenou prvním (interpolačním) dotazem v této fázi
vzhledem k počátku bloku. 
Srovnej $prv$ s očekávanou hodnotou $X$. 

$$|X-prv|\ge \frac{\hbox{počet dotazů v rámci bloku}-2}{2}\sqrt{delka}$$

protože když vynecháme první dva dotazy, tak se dále střídá binární dotaz
se skokem o $\sqrt{delka}$. Vynecháme-li i binární dotazy---vezmu každý
druhý---zůstanou jen skoky o $\sqrt{delka}$ a ty dohromady naskáčou méně
než je vzdálenost $x$ od prvního dotazu.

Označme $p_i=\pr(\hbox{v rámci bloku bylo položeno alespo\v n $i$ dotazů})$.
Pak jistě platí

$$\pr(\,|X-prv|\ge \frac{i-2}{2}\sqrt{delka})\ge p_i$$

Nyní použijeme Čebyševovu nerovnost, která říká, že
$$\pr(\,|X-EX|>t)\le \frac{\hbox{rozptyl }X}{t^2}$$

$$p_i \le \pr(\,|X-prv|\ge \frac{i-2}{2}\sqrt{delka})\le \frac{p(1-p)\;delka}
{(\frac{i-2}{2})^2 \;delka} \le \frac{1}{(i-2)^2}$$

protože $prv$ je očekávaná hodnota $X$ a $p(1-p)\le 1/4$ pro
$0\le p\le 1$. Celkem jsme dostali $p_i\le 1/(i-2)^2$.


Očekávaný čas pro práci v jednom bloku (pro jednu fázi) je $O(\hbox{očekávaný
počet dotazů v bloku})=O(\sum_{i=0}^\infty p_i)=O(\,3+\sum_{i=3}^\infty
1/(i-2)^2)=O(\,3+\pi^2/6)=O(4.6)$. To jsme pouze odhadli první tři členy
jedničkou a sečetli řadu, kterou asi znáte z analýzy.

Teď už snadno dopočítáme očekávaný čas zobecněného kvadratického
vyhledávání. Ten je $O(\,\hbox{(počet bloků)}\,\hbox{(očekávaný čas pro 1
blok)})=O(\,\log\log(|S|)\, O(1))=O(\,\log\log(|S|))$. Kde jsme vzali
počet bloků? Ten je určitě menší než počet dotazů v interpolačním
vyhledávání (jen interpolační dotazy).
