% Součást skript na Datové struktury. Viz main.tex
\markright{$ $Id$ $}

\def\prazdnatabulka{
\begin{tabular}{|l|l|}
\hline
& klíč\\
\hline
0& \\
1& \\
2& \\
3& \\
4& \\
5& \\
6& \\
7& \\
8& \\
9& \\
\hline
\end{tabular}
}

\chapter{Hašování I}

Je dáno universum $U$, reprezentovaná podmnožina $S, S \subseteq U, |S|=n$.
Velikost tabulky, ve které budeme chtít $S$ reprezentovat je $m$. 

Máme hašovací funkci $h: U \to \{0..m-1\}$. Množina $S$ je
reprezentována tabulkou $T[0..m-1]$ tak, že prvek $s \in S$ je uložen na
místě $h(s)$.

Charakteristickou vlastností obecné hašovací funkce je velké plýtvání 
pamětí pokud $n \ll |U|$, např. pro $n = \log\log |U|$.

S prvky, které nesou kladnou informaci ($x \in S$), moc
nenaděláme. Ale záporné můžeme nějak sloučit do jednoho nebo i
překrýt s těmi kladnými. To je základní idea hašování.


Problémy:
\begin{enumerate}
\item Jak rychle spočítáme $h(s)$.
\item Co znamená {\em uložen na místě $h(s)$} ? Co když platí $h(s)=h(t)$
a zároveň $s \ne t$ ?
\end{enumerate}

Řešení:
\begin{enumerate}
\item Omezíme se na rychle spočitatelné hašovací
funkce. Předpokládáme, že $h(s)$ spočteme v čase $O(1)$.
\item Tento případ se nazývá {\em kolize} a jednotlivé druhy hašování
se dělí podle toho, jak řeší kolize.
\end{enumerate}

% --------------------------------------------------------------------------
\section{Hašování se separovanými řetězci}

Tento způsob hašování řeší kolize tak, že kolidující prvky ukládá na
stejnou pozici do tabulky. Na této pozici jsou uloženy v podobě 
lineárních seznamů. Takovému seznamu kolidujících prvků se říká
\emph{řetězec}.
Hašovací tabulka je tedy pole lineárních seznamů, ne nutně uspořádaných.
Odtud plyne název této metody, protože řetězce prvků nejsou mezi sebou
promíchány - řetězec, který je v tabulce uložen na pozici $y$ v sobě 
obsahuje pouze ty prvky, pro které platí $h(x) = y$.

Základní operace na této tabulce jsou MEMBER (viz algoritmus
\ref{alg:hash1.separ.member}),
INSERT (viz alg. \ref{alg:hash1.separ.insert}) a DELETE (viz alg. 
\ref{alg:hash1.separ.delete}).

\begin{algorithm}[!htb]
\caption{MEMBER pro hašování se separovanými řetězci}
\label{alg:hash1.separ.member}
MEMBER(x):
\begin{enumerate}
\item Spočítáme $h(x)$.
\item Prohledáme $h(x)$-tý seznam.
\item Když tam $x$ je, vrátíme {\em true}, jinak {\em false}.
\end{enumerate}
\end{algorithm}


\begin{algorithm}[!htb]
\caption{INSERT pro hašování se separovanými řetězci}
\label{alg:hash1.separ.insert}
INSERT(x):
\begin{enumerate}
\item Spočítáme $h(x)$. {\it (Jako MEMBER)}
\item Prohledáme $h(x)$-tý seznam. {\it (Jako MEMBER)}
\item Když $x$ není v $h(x)$-tém seznamu, tak ho tam vložíme.
\end{enumerate}
\end{algorithm}


\begin{algorithm}[!htb]
\caption{DELETE pro hašování se separovanými řetězci}
\label{alg:hash1.separ.delete}
DELETE(x):
\begin{enumerate}
\item Spočítáme $h(x)$. {\it (Jako MEMBER)}
\item Prohledáme $h(x)$-tý seznam. {\it (Jako MEMBER)}
\item Když $x$ je v $h(x)$-tém seznamu, tak ho odstraníme.
\end{enumerate}
\end{algorithm}


Očekávaná doba operace MEMBER, INSERT nebo DELETE 
je stejná jako očekávaná délka seznamu.
Ale pozor na prázdný seznam, u něj nedosáhneme nulového času operace. 
Ukážeme, že očekávaná doba operace je konstantní.

\begin{samepage}
Předpoklady:
\begin{enumerate}
\item $h$ rozděluje prvky $U$ do seznamů nezávisle a rovnoměrně
(např. $h(x) = x \bmod m$).\\
Tedy pro $\forall {i,j}:{0 \leq i,j < m}$ se 
počty prvků $S$ zobrazených na $i$ a $j$ liší nejvýš o 1.
\item Množina $S$ má rovnoměrné rozdělení --- výběr konkrétní množiny
$S$ má stejnou pravděpodobnost. To je dost omezující, protože na
rozdíl od hašovací funkce nejsme schopni $S$ ovlivnit.
\end{enumerate}
\end{samepage}

% ..........................................................................
\subsection{Očekávaná délka seznamu}

Označme $p(\ell) = \pr(\hbox{seznam je dlouhý }\ell)$.

Z předpokladů má $p(\ell)$ binomické rozdělení, neboli
\[
p(\ell) = \rozvoj{n}{\ell}{\invm}
\]

\begin{samepage}
tedy očekávaná délka seznamu je
\begin{multline}\bigparens
\label{binom-uprava}
E 
= \sum_{\ell=0}^n \ell \cdot p(\ell)\\
= \sum_{\ell=0}^n \ell \frac{n!}{\ell!(n-\ell)!} (\invm)^\ell (1-\invm)^{n-\ell}
= \sum_{\ell=0}^n n \frac{(n-1)!}{(\ell-1)! [(n-1) - (\ell-1)]!} (\invm)^\ell (1-\invm)^{n-\ell}\\
= \sum_{\ell=0}^n \frac nm \binom{n-1}{\ell-1}(\invm)^{\ell-1} (1-\invm)^{(n-1)-(\ell-1)}
= \frac nm \sum_{k=-1}^{n-1} \binom{n-1}{k} (\invm)^k
(1-\invm)^{(n-1)-k}\\
\text{README.1st: všechny úpravy směřují k tomuto součtu podle binomické věty}\\
= \frac nm ( \invm + (1-\invm) )^{n-1}
= \frac nm
= \alpha,
\end{multline}

kde $\alpha = n/m$ je tzv.~\emph{faktor naplnění}\footnote{anglicky
\emph{load factor}}, obvykle je důležité, je-li větší či menší než 1.
\end{samepage}

\def\xx{
s rozptylem
{\bigparens $$ n \cdot \invm (1-\invm) $$}
}

% ..........................................................................
\subsection{Očekávaný čas posloupnosti operací}

Když máme posloupnost $P$ operací MEMBER, INSERT, DELETE splňující
předpoklad rovnoměrného rozdělení a aplikovanou na prázdnou hašovací
tabulku, pak očekávaný čas je \( O( |P| + \frac{|P|^2}{2m} ) \)

% ..........................................................................
\subsection{Očekávaný počet testů}

\mnote{co je to test? porovnání klíčů, nahlédnutí do tabulky?}
Složitost prohledání seznamu se může lišit podle toho, jestli tam hledaný
prvek je nebo není. \emph{Úspěšným případem} nazveme takovou Operaci(x), 
kde $x \in S$, \emph{neúspěšný případ} je $x \notin S$. V úspěšném případě
prohledáváme průměrně jenom polovinu seznamu.


Očekávaný čas pro neúspěšný případ 
\( \text{EČN} = O( {(1 - \invm)}^n + \frac nm ) \)

Očekávaný čas pro úspěšný případ 
\( \text{EČÚ} = O(\frac n{2m}) \)

\subsubsection{Neúspěšný případ}

Projdeme celý seznam, musíme nahlédnout i do prázdného seznamu.

$$ 
\text{EČN}
= 1 \cdot p(0) + \sum_{\ell=1}^n \ell \cdot p(\ell)
=         p(0) + \sum_{\ell=0}^n \ell \cdot p(\ell)
= (1-\invm)^n + \frac nm
\approx e^{-\alpha} + \alpha
$$

\subsubsection{Úspěšný případ}

\mnote{Zde Koubková 1998. Koubek 2000 to má trochu jinak}

Počet testů pro vyhledání všech prvků v seznamu délky $\ell$ je\\
$1+2+\cdots+\ell = \binom{\ell+1}{2}$.

\def\OcPocTst{\sum_\ell \binom{\ell+1}{2} p(\ell)}
Očekávaný počet testů je $ \OcPocTst $, 
očekávaný počet testů pro vyhledání všech prvků v tabulce je
$ m \cdot \OcPocTst $.

Ještě budeme potřebovat následující sumu, kterou spočítáme podobně
jako v \ref{binom-uprava}:
$$ \sum_{l=0}^n l^2 \rozvoj{n}{l}{\invm}
= \cdots =
\frac{n}{m} (1-\invm) + (\frac{n}{m})^2
$$

Očekávaný počet testů pro vyhledání jednoho prvku
\begin{multline}
\text{EČÚ}
= \frac{m}{n} \OcPocTst 
= \frac{m}{n} \cdot \frac{1}{2} ( \sum_\ell \ell^2 p(\ell) + \sum_\ell \ell \cdot p(\ell) )\\
= \frac{m}{2n} 
	( \frac{n}{m} (1-\invm)
	+ \frac{n^2}{m^2} 
	+ \frac{n}{m}
	)\\
= \frac{1}{2} - \frac{1}{2m} + \frac{n}{2m} + \frac{1}{2}
= 1 + \frac{n-1}{2m}\\
\sim 1 + \frac{\alpha}{2}
\end{multline}

% ..........................................................................
\subsection{Očekávaná délka nejdelšího seznamu}

\begin{samepage}
Známe očekávané hodnoty, ale ty nám samy o sobě moc neřeknou. Hodila
by se nám standardní odchylka, ta se ale složitě počítá. Místo toho
vypočteme očekávaný nejhorší případ:

Dokážeme, že
za předpokladů 1 a 2 a $|S| = n \leq m$ je očekávaná délka maximálního
seznamu $\text{EMS} = O( \frac{\log n}{\log\log n} )$.
\end{samepage}

Z definice 
\[
\text{EMS} = \sum_j j \cdot \pr(\text{maximální délka seznamu}=j). 
\]
Použijeme trik: nechť 
\[
q(j) = \pr(\text{existuje seznam, který má délku alespoň }j). 
\]
Pak 
\[ \pr(\text{maximální délka seznamu}=j) = q(j) - q(j+1) \]
a
\[
\text{EMS} = \sum_j q(j) 
\] (teleskopická suma) \mnote{vysvětlit}

Spočteme $q(j)$:
\[
q'(j) 
= \pr( \text{daný seznam má délku alespoň }j ) 
\leq \binom nj {(\frac 1m)}^j
\]
\[
q(j) \leq m \cdot q'(j)
\]
\[
\text{EMS}
\leq \sum \min(1, m \binom nj {(\frac 1m)}^j )
\leq \sum \min(1, m {(\frac nm)}^j \frac 1{j!} )
\leq \sum \min(1, \frac n{j!} )
\]

Nechť
\[
j_0 
=    \max\{k: k! \leq n \}
\leq \max\{k: (k/2)^{k/2} < n \}
= O( \frac{\log n}{\log\log n} ), 
\]
pak
\begin{multline}
\text{EMS}
\leq \sum_{j+0}^{j_0} 1 + \sum_{j=j_0}^\infty \frac n{j!}
=    j_0 + \sum_{j=j_0}^\infty \frac n{j_0!} \frac{j_0!}{j!}\\
\leq j_0 + \sum_{j=j_0}^\infty \frac{j_0!}{j!}
\leq j_0 + \sum_{j=j_0}^\infty (\frac 1{j_0})^{(j-j_0)} % geom řada
\leq j_0 + \frac 1{1 - 1/j_0}\\
= O(j_0) = O ( \frac{\log n}{\log\log n} )
\qed
\end{multline}


% --------------------------------------------------------------------------
\section{Hašování s uspořádanými řetězci}

Uspořádání řetězců vylepší neúspěšný případ, protože hledání v řetězci
můžeme zastavit v okamziku, kdy narazíme na prvek větší než hledaný prvek
(za předpokladu, že prvky v řetězci jsou uspořádány vzestupně).

\subsection{Očekávaný čas}

Očekávaný čas v neúspěšném případě se od času v úspěšném případě liší
jen o aditivní konstantu.

%(Jenom citace věty z \cite{Vitter-Chen})

% --------------------------------------------------------------------------
\section{Hašování s přesuny}

Zatím jsme předpokládali, že řetězce kolidujících prvků jsou uloženy
někde v dynamicky alokované paměti. 
To není výhodné, protože vyžaduje použití další paměti i když některé 
řetězce jsou prázdné.
Proto nyní budeme ukádat řetězce přímo v tabulce.

Řetězec na $i$-tém místě uložíme do tabulky tak, že 
první prvek je na $i$-tém místě a
pro každý prvek řetězce je v položce {\tt vpřed} adresa následujícího prvku
řetězce a v položce {\tt vzad} je adresa předchozího prvku. Začátek,
resp. konec řetězce má prázdnou položku {\tt vzad}, resp. {\tt vpřed}.

\begin{samepage}
\begin{priklad}
Například pro $U=\mathbb{N}, m=10, h(x) = x \bmod 10$, hašujeme posloupnost
10, 50, 21, 60: 

\vspace{5mm}
\begin{tabular}{|l|l|l|l|}
\hline
& klíč& {\tt vpřed}& {\tt vzad}\\
\hline
0& 10& 5& \\
1& 21& & \\
2& & & \\
3& 50& & 5\\
4& & & \\
5& 60& 3& 0\\
6& & & \\
7& & & \\
8& & & \\
9& & & \\
\hline
\end{tabular}
\end{priklad}
\end{samepage}
\vspace{5mm}

MEMBER(x) je jednoduchý - od $h(x)$-tého místa procházíme řetězec prvků až
na konec a v případě nalezení prvku operaci ukončíme.

Při INSERT musíme zjistit, zda $h(x)$-tý řetězec začíná na $h(x)$-tém
místě\footnote{To je jednoduché - pokud prvek na $h(x)$-tém místě má
nulový ukazatel {\tt vzad}, pak je začátkem $h(x)$-tého řetězce.}.
Pokud ano, prvek přidáme do $h(x)$-tého řetězce,
pokud ne, přemístíme prvek na $h(x)$-tém místě na jiné volné místo,
upravíme {\tt vpřed} a {\tt vzad} a prvek vložíme na $h(x)$-té místo.

% vynuceny zlom stranky, nechtel si dat rict pomoci samepage
\pagebreak

\begin{samepage}
\begin{priklad}
Tabulka z předchozího příkladu po INSERT(53) bude vypadat takto: 

\vspace{5mm}
\begin{tabular}{|l|l|l|l|}
\hline
& klíč& {\tt vpřed}& {\tt vzad}\\
\hline
0& 10& 5& \\
1& 21& & \\
2& & & \\
3& 53& & \\
4& 50& & 5\\
5& 60& 4& 0\\
6& & & \\
7& & & \\
8& & & \\
9& & & \\
\hline
\end{tabular}
\end{priklad}
\end{samepage}
\vspace{5mm}

Při DELETE musíme testovat, zda odstraňovaný prvek není na 1. místě
svého řetězce
a pokud ano a řetězec má více prvků, musíme přesunout jiný prvek z
tohoto řetězce na místo odstraňovaného prvku.

Jak zjistíme, že jiný prvek $y$ patří tam, kde je uložen? 
\mnote{Tady mám zmatek. Zavést lepší značení.}
Spočítat $h(y)$
může být relativně pomalé. Pokud {\tt T[i].vzad} někam ukazuje, pak
víme, že $h(y) \ne h(x)$.


% --------------------------------------------------------------------------
\section{Hašování se dvěma ukazateli}

Při hašování s přesuny ztrácíme čas právě těmi přesuny, obzvláště,
když jsou záznamy velké. To motivuje následující implementaci 
hašování s řetězci.

Použijeme dva ukazatele, {\tt vpřed} a {\tt začátek}:
\begin{itemize}
  \item {\tt T[i].vpřed} je index následujícího prvku v řetězci, 
  	který je zde uložen. (Nemusí to být řetězec s $h(x)=i$.)
  \item {\tt T[i].začátek} je index začátku řetězce, který obsahuje
  	prvky, jejichž $h(x)=i$.
\end{itemize}

\begin{priklad}
Nechť $h(x) = x \bmod 10$. Ukládáme prvky 50, 90, 31, 60:

\vspace{5mm}

\begin{tabular}{|l|l|l|l|}
\hline
& klíč& vpřed& začátek\\
\hline
0& 50& 3& 0\\
1& 31& & 1\\
2& 60& & \\
3& 90& 2& \\
4& & & \\
5& & & \\
6& & & \\
7& & & \\
8& & & \\
9& & & \\
\hline
\end{tabular}

\vspace{8mm}

% opet, samepage nefunguje
\pagebreak

\begin{samepage}
Přidáme 42, 72, 45:

\vspace{5mm}

\begin{tabular}{|l|l|l|l|}
\hline
& klíč& vpřed& začátek\\
\hline
0& 50& 3& 0\\
1& 31& & 1\\
2& 60& & 5\\
3& 90& 2& \\
4& 45& & \\
5& 42& 6& 4\\
6& 72& & \\
7& & & \\
8& & & \\
9& & & \\
\hline
\end{tabular}
\end{samepage}
\end{priklad}


% --------------------------------------------------------------------------
\section{Hašování s lineárním přidáváním}

Jde to i bez ukazatelů.

Je dáno $m$ míst, která tvoří tabulku. Pokud je příslušné políčko již
zaplněno, hledáme cyklicky první volné následující místo a tam
zapíšeme. Vhodné pro málo zaplněnou tabulku ($<60\%$, pro $80\%$ 
vyžaduje už hodně času). Téměř nemožné DELETE: buď označit místo jako smazané, nebo
celé přehašovat. 

%\subsection{Očekávaný počet testů}
%(Jenom citace věty z Knutha, dk na cvičení)

\vspace{5mm}

\begin{tabular}{|l|l|}
\hline
& klíč\\
\hline
0& 120\\
1& 51\\
2& 72\\
3& \\
4& \\
5& \\
6& \\
7& \\
8& \\
9& \\
\hline
\end{tabular}

\vspace{8mm}

Přidáme 40, 98, 62, 108:

\vspace{5mm}

\begin{tabular}{|l|l|}
\hline
& klíč\\
\hline
0& 120\\
1& 51\\
2& 72\\
3& 40\\
4& 62\\
5& \\
6& \\
7& \\
8& 98\\
9& 108\\
\hline
\end{tabular}


% --------------------------------------------------------------------------
\section{Hašování se dvěma funkcemi (otevřené h., h. s otevřenou adresací)}

Použijeme dvě hašovací funkce, $h_1$ a $h_2$, je zde však účelné předpokládat, 
že $h_2(i)$ a $m$ jsou nesoudělné pro každé $i\in U$. Při INSERTu pak hledáme
nejmenší $j$ takové, že $T[h_1(x) + j h_2(x)]$ je volné.

\begin{priklad}
Mějme $h_1(x) = x \bmod 10$

\vspace{5mm}

\begin{tabular}{|l|l|}
\hline
& klíč\\
\hline
0& 10\\
1& 31\\
2& \\
3& \\
4& \\
5& \\
6& \\
7& \\
8& \\
9& \\
\hline
\end{tabular}

\vspace{5mm}

INSERT(100): $h_1(100)=0$ a předpokládejme, že $h_2(100)=3$. Volné
místo najdeme pro $i=1$.

INSERT(70): $h_1(70)=0$ a předpokládejme, že $h_2(70)=1$. Volné
místo najdeme pro $i=2$.

\vspace{5mm}

\begin{tabular}{|l|l|}
\hline
& klíč\\
\hline
0& 10\\
1& 31\\
2& 70\\
3& 100\\
4& \\
5& \\
6& \\
7& \\
8& \\
9& \\
\hline
\end{tabular}
\end{priklad}

\vspace{5mm}

Neuvedli jsme explicitní vzorec pro $h_2$. Její sestavení je totiž
složitější. Všimněte si, že nemůžeme vzít $h_2(100)=4$. Všechny
hodnoty $h_2$ totiž musí být nesoudělné s velikostí tabulky.

\def\xx{
Např.: $h(i) = i \bmod 10$, \( h_1(i) = \begin{cases}
1 &\text{když}\ i \equiv 1 \pmod 3 \\
3 &\text{když}\ i \equiv 0 \pmod 3 \\
7 &\text{když}\ i \equiv 2 \pmod 3
\end{cases}
\)
}

\subsection{Algoritmus INSERT}
% \mnote{Ještě rozmyslet značení a sjednotit zápis algoritmů}

viz algoritmus \ref{alg:hash1.2fce.insert}.

\begin{algorithm}[!htb]
\caption{INSERT pro hašování se dvěma funkcemi}
\label{alg:hash1.2fce.insert}
INSERT(x)
\begin{enumerate}
\item 
 spočti $i=h_1(x)$
\item
 když tam $x$ je, pak skonči\\
 když je místo prázdné, vlož tam $x$ a skonči
\item 
 \begin{algorithmic}
 \IF {$i$-té místo obsazeno prvkem $\ne x$}
 \STATE spočti $h_2(x)$
 \STATE $k = (h_1(x)+h_2(x)) \bmod m$
 \WHILE {$k$-té místo je obsazené prvkem $\ne x$ a $i\ne k$}
   \STATE \quad $k = (k+h_2(x)) \bmod m$
 \ENDWHILE
 \ENDIF
 \end{algorithmic}
\item
 když je $k$-té místo obsazené prvkem $x$, pak nedělej nic,\\ 
 když $i=k$, pak ohlaš přeplněno, jinak vlož $x$ na
 $k$-té místo
\end{enumerate}
\end{algorithm}


Test $k\ne i$ v kroku 3 brání zacyklení algoritmu. Tento problém má 
alternativní řešení, nedovolíme vložení posledního prvku (místo testu v  
cyklu si pamatujeme údaje navíc). Analogické problémy nastávají 
u hašování s lineárním přidáváním. 

Tato metoda pracuje dobře až do 90\% zaplnění.

\subsection{Očekávaný počet testů}

Předpokládáme, že posloupnost $h_1(x), h_1(x)+h_2(x), h_1(x)+2h_2(x), \dots$
je náhodná, tedy že pro každé $x$ mají všechny permutace řádků tabulky stejnou
pravděpodobnost, že se stanou touto posloupností.

\subsubsection{při neúspěšném vyhledávání}

Označíme jej $C(n,m)$, kde $n$ je velikost reprezentované množiny a
$m$ je velikost hašovací tabulky.

Buď $q_j(n,m)$ pravděpodobnost, že v tabulce velikosti $m$ s uloženou
množinou velikosti $n$ jsou při INSERT(x) obsazená místa
$h_1(x)+ih_2(x)$ pro $i=0..j-1$ (tedy řetězec má alespoň $j$ prvků).

\begin{multline}
C(n,m)	= \sum_{j=0}^n (j+1)(q_j(n,m) - q_{j+1}(n,m) \\
	=(\sum_{j=0}^n q_j(n,m) ) - (n+1) q_{n+1}(n,m)
	= \sum_{j=0}^n q_j(n,m)
\end{multline}
Protože
\begin{equation}
q_j(n,m) = \frac nm \frac{n-1}{m-1} \cdots \frac{n-j+1}{m-j+1}
	=  \frac nm q_{j-1}(n-1, m-1)
\end{equation}

dostaneme po dosazení:
\begin{multline}
\ldots	= 1 + \sum_{j=1}^\infty \frac nm q_{j-1}(n-1, m-1)
	= 1 + \frac nm \sum_{j=1}^\infty q_{j-1}(n-1, m-1) \\
	= 1 + \frac nm C(n-1, m-1)
\end{multline}

Řešením tohoto rekurentního vzorce je

\begin{equation}
C(n,m) = \frac{m+1}{m-n+1}, 
\end{equation}
což dokážeme indukcí:
\begin{multline}
C(n,m)	= 1 + \frac nm C(n-1, m-1) \\
	= 1 + \frac nm \frac{m}{m-n+1} = \frac{m-n+1+n}{m-n+1} 
	= \frac{m+1}{m-n+1}
\approx   \frac 1{1-\alpha}
\end{multline}

\subsubsection{při úspěšném vyhledávání}

Součet očekávaných testů všech INSERTů přes vytváření reprezentované
množiny dělený velikostí množiny.
\begin{multline}
= \frac 1n	\sum_{i=0}^{n-1} C(i, m)
= \frac 1n	\sum_{i=0}^{n-1} \frac{m+1}{m-i+1}
= \frac {m+1}n	\sum_{i=0}^{n-1} \frac    1{m-i+1} \\
= \frac {m+1}n
	((\sum_{i=1}^{m+1} \frac 1i) - (\sum_{i=1}^{m-n+1} \frac 1i))
\approx \frac {m+1}n \ln \frac{m+1}{m-n+1}
\approx \frac 1\alpha \ln \frac 1{1-\alpha}
\end{multline}

Následující tabulka dává očekávanou dobu vyhledávání pro různé 
zaplnění hašovací tabulky.

\vspace{5mm}

\centerline{\begin{tabular}{c|cccccc}
$\alpha$ 
 & 0.5 & 0.7 & 0.9 & 0.95 & 0.99 & 0.999 \\
\hline
$\frac 1{1-\alpha}$ 
 & 2   & 3.3 & 10  & 20   & 100  & 1000 \\
$\frac 1\alpha \ln \frac 1{1-\alpha}$
 & 1.38& 1.7 & 2.55& 3.15 & 4.65 & 6.9
\end{tabular}}

% --------------------------------------------------------------------------
\section{Srůstající hašování (standardní: LISCH a EISCH)}

\mnote{XXX doplnit odhady pro úspěšný/neúspěšný případ pro EISCH, LISCH, ..
z 2003/2004.}

Další metodou jak řešit kolize je neseparovat řetězce. To znamená, že v
jednom řetězci se nyní mohou nacházet prvky, které mají různé hodnoty
hašovací funkce. Proto se této metodě říká \emph{srůstající hašování}.
Řetězce se nepřesouvají, v tabulce tedy nemusí být obsažena informace 
kde začíná daný řetězec.

Tabulka má dvě položky: klíč a adresu následujícího prvku ({\tt vpřed}). 
Prvek $s \in S$ je reprezentován v řetězci, který pokračuje v místě $h(s)$.

\begin{priklad}
\label{priklad:srust.hash}
Opět použijeme hašovací funkci $h(x) = x \mod 10$.
V následující tabulce jsou srostlé řetězce pro hodnoty 0 a 3 funkce $h$:

\vspace{5mm}

\begin{tabular}{|l|l|l|}
\hline
& klíč& {\tt vpřed}\\
\hline
0& 10& 3\\
1& 21& \\
2& & \\
3& 40& 4\\
4& 33& 7\\
5& & \\
6& & \\
7& 70& \\
8& & \\
9& & \\
\hline
\end{tabular}
\end{priklad}

\vspace{5mm}

\subsection{Algoritmus INSERT}

viz algoritmus \ref{alg:hash1.srust.insert}

\begin{algorithm}[!htb]
\caption{INSERT pro srůstající hašování}
\label{alg:hash1.srust.insert}
INSERT(x)
\begin{enumerate}
\item
 spočti $i=h(x)$
\item
 prohledej řetězec začínající na místě $i$ a pokud nenajdeš $x$,
 přidej ho do tabulky a připoj ho do toho řetězce.
\end{enumerate}
\end{algorithm}

Kam do toho řetězce máme připojit nový prvek? (To je jiná otázka, než
které volné místo zvolit.) Metoda LISCH (\emph{late insert standard 
coalesced hashing}) ho připojí na poslední místo řetězce, 
metoda EISCH (\emph{early insert standard coalesced hashing}) 
ho připojí hned za $h(x)$-té místo.

\begin{priklad}
Po provedení operací
LISCH INSERT(103), EISCH INSERT(94) bude tabulka z příkladu
\ref{priklad:srust.hash} vypadat takto:

\vspace{5mm}

\begin{tabular}{|l|l|l|}
\hline
& klíč& {\tt vpřed}\\
\hline
0& 10& 3\\
1& 21& \\
2& & \\
3& 40& 4\\
4& 33& 6\\
5& & \\
6& 94& 7\\
7& 70& 9\\
8& & \\
9& 103& \\
\hline
\end{tabular}
\end{priklad}

\vspace{5mm}

\begin{pozn}
Při úspěšném vyhledání je EISCH asi o 15\% rychlejší než LISCH. (Při
neúspěšném jsou samozřejmě shodné, protože v obou případech je nutné 
projít celý řetězec.)
\end{pozn}

% XXX co s timto ?
\def\xx{
\subsection{Algoritmus MEMBER}
\subsection{Algoritmus INSERT LISCH}
\subsection{Algoritmus INSERT EISCH}
\subsection{Očekávaný počet testů}
(Jenom citace věty z~\cite{Vitter-Chen})
}

% --------------------------------------------------------------------------
\section{Srůstající hašování s pomocnou pamětí (obecné: LICH, EICH, VICH)}

Standardní srůstající hašování má tu nevýhodu, že se při větším
zaplnění tabulky mohou vytvořit dlouhé řetězce. Tabulku tedy prodloužíme o
% tady bylo {\uv sklep} ale nejak to nefunguje XXX
pomocnou pamět (tzv. \emph{sklep}), do které se nedostane hašovací funkce, a
kolidující prvky přidáváme odspodu. Řetězce tedy srostou až po
zaplnění sklepa.

Opět existují varianty připojení nového prvku do řetězce: LICH a EICH
jsou analogické k LISCH a EISCH. VICH (\emph{variable insert coalesced
hashing}) připojuje na konec řetězce, jestliže řetězec končí ve sklepě,
jinak na místo, kde řetězec opustil sklep.

% opet, samepage nefunguje
\pagebreak

\begin{samepage}
\begin{priklad}
Provedeme operaci INSERT pro následující posloupnost prvků:
10, 41, 60, 70, 71, 90, 69, 40, 79. Tabulka pak bude vypadat takto:

% XXX \newcommand{\x}[1]{(#1)}
\vspace{5mm}

\begin{tabular}{|l||l|l||l|l||l|l|}
\hline
 & \multicolumn{2}{|c||}{LICH}& 
   \multicolumn{2}{|c||}{EICH}&
   \multicolumn{2}{|c|}{VICH}\\
\hline
 & klíč& {\tt vpřed} & klíč& {\tt vpřed} & klíč& {\tt vpřed}\\
\hline
0&	10& 12&		10& (12)(11)(9)7& 10& 12\\
1&	41& 10&		41& 10&		41& 10\\
2&	& &		& &		& \\
3&	& &		& &		& \\
4&	& &		& &		& \\
5&	& &		& &		& \\
6&	79& &		79& 8&		79& 8\\
7&	40& 6&		40& 9&		40& 9\\
8&	69& 7&		69& 11&		69& \\
9&	90& 8&		90& (11)(8)6&	90& (8)6\\
\hline
10&	71& &		71& &		71& \\
11&	70& 9&		70& 12&		70& (9)7\\
12&	60& 11&		60& &		60& 11\\
\hline
\end{tabular}

\vspace{5mm}

Sklep je od "hlavní části" tabulky odlišen čarou.
\end{priklad}
\end{samepage}

\def\xx{%commented out
\begin{tabular}{|l|l|l|}
\hline
& klíč& {\tt vpřed}\\
\hline
0& 10& 10\\
1& 41& 12\\
2& 90& 3\\
3& 62& \\
4& & \\
5& & \\
6& & \\
7& & \\
8& & \\
9& & \\
\hline
10& 60& 11\\
11& 70& 2\\
12& 71& \\
\hline
\end{tabular}
}%commented out


\subsection{Operace DELETE}

viz algoritmus \ref{alg:hash1.srust.pomoc.delete}

\begin{algorithm}[!htb]
\caption{DELETE pro srůstající hašování s pomocnou pamětí}
\label{alg:hash1.srust.pomoc.delete}
\begin{enumerate}
\item spočítáme $h(x)$ a prohledáme řetězec začínající na adrese $h(x)$. \\
když neobsahuje $x$ $\rightarrow$ konec.
\item když $x$ je na adrese $h(x)$ a v řetězci následuje prvek v pomocné části
tabulky (sklepě) - odstraníme $x$. Prvek, který následuje za $x$ přesuneme 
na místo $h(x)$ a uvolníme jeho místo - konec.
\item $x$ je ve sklepě - zjístíme, zda se nachází v řetězci mimo sklep prvek
s hašovací hodnotou $h(x)$ - pokud neexistuje, přesuneme poslední prvek v
řetězci, který je ve sklepě na místo $x$ a místo posledního prvku uvolníme.
\par
Když takový prvek existuje, vezmeme první prvek s touto vlastností.
Označme ho $y$, pak $y$ přeneseme na místo $x$ a budeme chtít uvolnit místo
prvku $y$. (obrazně řečeno $x$ a $y$ v řetězci prohodíme)
\item $x$ je v části, kam může hašovací funkce a řetezec v této části
pokračuje. pak $x$ odstraníme a prvky za $x$ zahašujeme do tabulky v pořadí
jak se do ní ukládaly a pokud vzniká kolize, umístíme je zpátky na místo,
kde byly.
\end{enumerate}
\end{algorithm}

% chci aby se priklad vlezl na jednu stranu
\pagebreak

\begin{priklad}
Použijeme hašovací funkci $h(x) = x \mod 10$

Provedeme operaci DELETE na prvky 41, 42.
\mnote{? 42 v puv. tabulce nebylo XXX sehnat puvodni priklad}

\begin{itemize}
\item metoda LICH: \\
odstraníme prvky 62,78,82,52 z řetězce a provedeme postupně 
INSERT: 62,78,82,52

\item metoda VICH: \\
odstraníme 62,78,82,52 a provedeme INSERT: 82,62,78,52.
pokud bychom provedli INSERT 82,62,52,78, pak bychom nepoznali, v jakém to
bylo v tabulce pořadí.
\end{itemize}

\vspace{5mm}

\begin{tabular}{|l||l|l|}
\hline
 & klíč& {\tt vpřed} \\
\hline
0&	& \\
1&	11& 10\\
2&	32& 9\\
3&	43& \\
4&	& \\
5&	42& 9\\
6&	82& \\
7&	78& 6\\
8&	62& 7\\
9&	52& 8\\
\hline
10&	21& 11\\
11&	41& 12\\
12&	61& \\
\hline
\end{tabular}
\end{priklad}

\vspace{5mm}


\subsection{Srovnávací graf}

XXX
\mnote{Nascanovat obrázek z Vittera a Chena~\cite{Vitter-Chen}}


% --------------------------------------------------------------------------
\section{Srovnání metod}

Zde uvádíme porovnání podle počtu testů, protože to se dá
{\em vypočítat}. Doba běhu se musí {\em měřit}.

\subsection{Pořadí podle neúspěšného případu}

\begin{samepage}
V neúspěšném případě:
\begin{enumerate}
\item h. s uspořádanými řetězci (nejlepší)
\item h. s přesuny
\item VICH=LICH a h. se 2 ukazateli (VICH je lepší až do $\alpha=3/4$)
\item EICH
\item LISCH=EISCH (až sem je vše $O(1)$)
\item h. se 2 funkcemi
\item h. s lineárním přidáváním (nejhorší)
\end{enumerate}
\end{samepage}


\subsection{Počet testů pro zaplněnou tabulku}

Počet testů pro úplně zaplněnou tabulku ($m=n$ nebo $m=n-1$)

\vspace{5mm}

\mnote{co znamenají čísla v pravém sloupci ?}
\begin{tabular}{|l|l|}
\hline
typ	& XXX \\
\hline
h. s přesuny&		1.5\\
h. se 2 ukazateli&	1.6\\
VICH=LICH&		1.8\\
EICH&			1.92\\
LISCH=EISCH&		2.1\\
\hline
h. se 2 funkcemi&	$n$\\
h. s lineárním přidáváním&$n$\\
\hline
\end{tabular}

%\vspace{5mm}
%\bigskip

\subsection{Počet testů vyjádřený vzorcem}

\begin{tabular}{|l|l|l|}
\hline
typ &úspěšné vyhledání &neúspěšné hledání\\
\hline
s řetězci &$ 1+\frac\alpha2 $ &$e^{-\alpha} + \alpha$\\
s uspořádanými řetězci &$ 1+\frac\alpha2 $ &
	$e^{-\alpha} + 1 + \frac\alpha2 - \frac1\alpha (1-e^\alpha)$\\
s přemisťováním &$ 1+\frac\alpha2 $ &$e^{-\alpha} + \alpha$\\
se 2 ukazateli &$ 1 + \frac\alpha2 + \frac{\alpha^2}2 $ &
	$1 + \frac{\alpha^2}2 + \alpha + e^{-\alpha}(2+\alpha) - 2$\\
s lineárním přidáváním &$\frac12 (1 + \frac1{1-\alpha})$ &
	$\frac12 (1 + {(\frac1{1-\alpha})}^2 )$\\
dvojité hašování &$\frac 1\alpha \ln \frac 1{1-\alpha}$ &$\frac 1{1-\alpha}$\\
LISCH &\(\begin{aligned}
	 1 + \frac 1{8\alpha} (e^{2\alpha}-1-2\alpha)\\ + \frac14 \alpha
	\end{aligned}
	\)&
	$1+ \frac14 (e^{2\alpha}-1-2\alpha)$\\
EISCH &$\frac 1\alpha (e^\alpha - 1)$&
	$1+ \frac14 (e^{2\alpha}-1-2\alpha)$\\
\hline
\end{tabular}

\vspace{10mm}

% mimo tabulku, jinak to je hnus
LICH úspěšné: 
 \(
 \begin{cases}
  1+ \frac \alpha{2\beta}
	&\text{když } \alpha \leq \lambda\beta\\
  1
  +\frac\beta{8\alpha}(e^{2(\alpha/\beta-\lambda)}-1-2(\alpha/\beta-\lambda))\\
  \qquad \times (3 - \frac2\beta + 2\lambda)\\
  \quad + \frac14 (\frac\alpha\beta + \lambda)
  + \frac\lambda4 (1 - \frac{\lambda\beta}\alpha)
	&\text{když } \alpha \geq \lambda\beta
 \end{cases}
 \)

LICH neúspěšné: 
 \(
 \begin{cases}
	e^{-\alpha/\beta} + \frac\alpha\beta 
		&\text{když } \alpha \leq \lambda\beta\\
	\frac1\beta
	+ \frac14 (e^{2(\alpha/\beta-\lambda)}-1)(3 - \frac2\beta + 2\lambda)\\
	\quad - \frac12 (\alpha/\beta - \lambda)
		&\text{když } \alpha \geq \lambda\beta 
 \end{cases}
 \),

kde $\beta = \frac m{m'}$ je podíl adresové části tabulky na její
celkové velikosti
a $\lambda$ je jediné nezáporné řešení rovnice 
$e^{-\lambda} + \lambda = \frac1\beta$.


% --------------------------------------------------------------------------
\section{Přehašovávání}

V předchozích metodách jsme narazili na případy, že při velkém
zaplnění tabulky je nutné ji přehašovat. Zde si ukážeme metodu, jak se to
dělá.

Máme hašovací funkce: $h_0$ hašuje do tabulky velikosti $m = 2^0 m$,
$h_1$ do $2m = 2^1 m$, $h_2$ do $4m = 2^2 m$ \ldots, $h_i$ do $2^i
m$.

% opet, samepage nezafungovalo
%\pagebreak
\vspace{10mm}

\begin{samepage}
\noindent
Množinu $S$ reprezentujeme takto:

Uložena je velikost $S$ a číslo $i$ takové, že
\begin{equation}
\label{prehas}
2^{i-2} m < |S| < 2^i m
\end{equation}
a S je zahašována funkcí $h_i$.
\end{samepage}

\subsection{MEMBER}

MEMBER funguje normálně, při INSERT a DELETE kontrolujeme porušení
podmínky \eqref{prehas} a případně přehašujeme pro $i \pm 1$:

\subsection{INSERT}

Provedeme operaci INSERT a když máme přidat prvek, testujeme,
zda $|S|+1 < 2^i m$. Pokud nerovnost platí, dokončíme INSERT. Pokud
neplatí, zvětšíme $i$ o 1 a spočítáme uložení $S \cup \{x\}$ vzhledem
k nové hašovací funkci $h_i$.

\subsection{DELETE}

Provedeme operaci DELETE a když máme odstranit prvek, testujeme,
zda $i>0$ a $|S|-1 = 2^{i-2} m$. Pokud rovnost neplatí, dokončíme DELETE. Pokud
platí, zmenšíme $i$ o 1 a spočítáme uložení $S - \{x\}$ vzhledem
k nové hašovací funkci $h_i$.

\subsection{Složitost}

Tato metoda má malou amortizovanou složitost. Když se spočítá hašovací 
tabulka pro novou hašovací funkci $h_i$, pak obsahuje $2^{i-1}m$ prvků a 
proto je třeba alespoň $2^{i-2}m$ úspěšných operací DELETE nebo $2^{i-1}m$ 
úspěšných operací INSERT, abychom přepočítávali tabulku pro jinou hašovací 
funkci. Tedy amortizovaná složitost přepočítávání tabulky je $O(1)$ (tato 
metoda není vhodná pro práci v interaktivním režimu).

