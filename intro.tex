% Součást skript na Datové struktury. Viz main.tex
\markright{$ $Id$ $}

\chapter{Úvod}

Chceme reprezentovat data, provádět s nimi operace. Cílem této
přednášky je popsat ideje, jak datové struktury reprezentovat, popsat
algoritmy pracující s nimi a přesvědčit vás, že když s nimi budete
pracovat, měli byste si ověřit, jak jsou efektivní.

Problém měření efektivity: většinou nemáme šanci vyzkoušet všechny
případy vstupních dat. Musíme buď doufat, že naše vzorky jsou
dostatečně reprezentativní, nebo to vypočítat. Tehdy ale zase nemusíme
dostat přesné výsledky, pouze odhady.

% --------------------------------------------------------------------------
\section{Předpoklady}

\begin{enumerate}
\item Datové struktury jsou nezávislé na řešeném problému;
abstrahujeme. Například u slovníkových operací {\it vyhledej, přidej,
vyjmi,\/} nás nezajímá, jestli slovník reprezentuje body v prostoru,
vrcholy grafu nebo záznamy v databázi.
\item V programu, který řeší nějaký problém, se příslušné datové
struktury používají {\em velmi často}.
\end{enumerate}

% --------------------------------------------------------------------------
\section{Jaké typy složitosti nás zajímají}

\subsection{Paměťová složitost reprezentované struktury}
Je důležitá, ale obvykle jednoduchá na spočítání a není šance ji
vylepšit --- jedině použít úplně jinou strukturu. Proto ji často
nebudeme ani zmiňovat.

\subsection{Časová složitost algoritmů\\ pracujících na datové struktuře}

\subsubsection{Časová složitost v nejhorším případě}
Její znalost nám zaručí, že nemůžeme být nepříjemně překvapeni (dobou
běhu algoritmu). Hodí se pro {\em interaktivní} režim --- uživatel
sedící u databáze průměrně dobrou odezvu neocení, ale jediný pomalý
případ si zapamatuje a bude si stěžovat. Za vylepšení nejhoršího
případu obvykle platíme zhoršením průměrného případu.

\subsubsection{Očekávaná časová složitost}
Je to vlastně vážený průměr --- složitost každého případu vstupních
dat násobíme pravděpodobností jeho výskytu a sečteme. Je zajímavá pro
dávkový režim zpracování. Například Quicksort patří mezi nejrychlejší známé
třídící algoritmy, ale v nejhorším případě má složitost kvadratickou.

Pozor na předpoklady o rozdělení vstupních dat. Je známý fakt, že pro každé 
$k$ existuje číslo $n_k$ takové že každý náhodný graf s alespoň $n_k$ vrcholy
s velkou pravděpodobností obsahuje kliku velikosti $k$. To vede k 
následujícímu algoritmu, který určí zda graf je nejvýše $k-1$ barevný s 
očekávaným konstantním časem.

Algoritmus: vstup graf $(V,E)$.
\begin{enumerate}
\item
Když velikost $V$ je menší než $n_k$, pak prohledáním všech možností 
urči barevnost grafu $(V,E)$ a vydej odpověď, jinak pokračuj na následující 
bod.
\item
Zvol náhodně $n_k$ vrcholů v množině $V$ a zjisti zda indukovaný podgraf 
na této množině obsahuje kliku velikosti $k$. Pokud ano, odpověď je ne, 
jinak pokračuj na následující bod.
\item
Prohledáním všech možností urči barevnost grafu $(V,E)$ a vydej odpověď.
\end{enumerate}


Tento algoritmus se ale pro praktické použití moc nehodí, protože normálně 
se například s náhodnými grafy na 200 vrcholech nesetkáváme.

\subsubsection{Amortizovaná složitost}
Pro každé $n$ nalezneme nejhorší čas vyžadovaný posloupností $n$ 
operací a tento čas vydělíme $n$. Amortizovaná složitost je limitou 
těchto hodnot pro $n$ jdoucí do nekonečna. To nás zajímá proto, že některé
datové struktury mají takovou vnitřní organizaci, že na ní závisí
složitost, a ta organizovanost se během posloupnosti operací
mění. Nejhorší případ vlastně uklízí za následující nebo
předchozí rychlé případy.

Například u přičítání jedničky ke $k$-cifernému binárnímu číslu je
časová složitost počet jedniček ve vstupu. Nejhorší případ s lineární
složitostí nastane pro číslo ze samých jedniček (tedy $2^k-1$), ale
těch případů je málo a amortizovaná složitost nakonec vyjde
konstantní.

Nebo určité algoritmy v překladačích v praxi běží rychle, přestože
jednotlivé operace mají velkou složitost v nejhorším případě. 
To se také podařilo vysvětlit pomocí amortizované složitosti.

\subsubsection{Asymptotická notace}

Skutečná složitost závisí na implementaci algoritmu, na konkrétním
počítači, vlastně se nedá přesně spočítat v obecném případě. Abychom 
mohli spočítat aspoň něco, začaly se používat odhady složitosti až 
na multiplikativní konstantu. Tyto odhady popisují růst složitosti 
vzhledem ke zvětšujícím se vstupům, ale neurčují konkrétní funkci 
(čísla). 

Nechť $f, g$ jsou funkce na přirozených číslech. Značíme:
\begin{eqnarray}
f = O(g),	&\text{když }
	&\exists c>0 \ \forall n: f(n) \leq c g(n) \\
f = \omega(g)	&
	&\forall c>0 \ \exists \text{nekonečně mnoho }n: f(n) > c g(n)\\
f = \Theta(g)	&
	&\exists c,d>0 \ \forall n: d g(n) \leq f(n) \leq c g(n)
\end{eqnarray}

Budeme převážně používat $O$, protože chceme hlavně horní odhady, kdežto dolní
odhady bývá obvykle těžší zjistit. Pro úplnost:
\[ f=o(g), \text{když } \lim_{n \to \infty} \frac{f(n)}{g(n)} = 0 \]


% ==========================================================================
\chapter{Slovníkový problém}

Je dáno universum $U$, máme reprezentovat jeho podmnožinu $S \subseteq
U$.

\newcommand{\MEMBER}{\text{MEMBER}}
\newcommand{\INSERT}{\text{INSERT}}
\newcommand{\DELETE}{\text{DELETE}}
Budeme používat operace
\begin{itemize}
\item $\MEMBER(x), x \in U,$ 
	odpověď je {\em ano,} když $x \in S$, {\em ne,} když $x \notin S$.
\item $\INSERT(x), x \in U,$ 
	vytvoří reprezentaci množiny $S \cup \{x\}$
\item $\DELETE(x), x \in U,$ 
	vytvoří reprezentaci množiny $S \setminus \{x\}$
\item ACCESS($x$). Ve skutečných databázích MEMBER nestačí, protože se
kromě klíče prvku zajímáme i o jeho ostatní atributy. Tady se ale o ně
starat nebudeme --- obvyklé řešení je mít u klíče pouze ukazatel na
ostatní data, což usnadňuje přemísťovaní jednotlivých prvků datové struktury.
\end{itemize}

Předpokládá se znalost těchto základních datových struktur: pole,
spojový seznam, obousměný seznam, zásobník, fronta, strom.

% --------------------------------------------------------------------------
\section{Pole}

Do pole velikosti $|U|$ uložíme charakteristickou funkci $S$.
\begin{itemize}
\item[+] Velmi jednoduché a rychlé --- všechny operace probíhají v
konstantním čase $O(1)$
\item[--] Paměťová náročnost $O(|U|)$, což je kámen
úrazu. Např. databáze všech lidí v Česku, kódovaných rodným číslem, by
snadno přerostla možnosti 32-bitového adresního prostoru (10 miliard
RČ $\times$ 4B ukazatel) \mnote{Najít lepší příklad?} Ale pro grafové
algoritmy je tahle reprezentace velmi vhodná.
\end{itemize}

% --------------------------------------------------------------------------
\section{Seznam}

Vytvoříme seznam prvků v $S$, operace provádíme prohledáním
seznamu. Časová i paměťová složitost operací je $O(|S|)$ (a to i pro
INSERT --- musíme zjistit, jestli tam ten prvek už není).
